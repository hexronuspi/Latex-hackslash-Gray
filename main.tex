\documentclass{article}
\usepackage{graphicx}
\usepackage{tcolorbox}
\usepackage{listings}
\usepackage{color}
\usepackage{xcolor}
\usepackage{minted}
\usepackage[rightcaption]{sidecap}

\usepackage{wrapfig}

\usepackage{hyperref}

\definecolor{codegreen}{rgb}{0,0.6,0}
\definecolor{codegray}{rgb}{0.5,0.5,0.5}
\definecolor{codepurple}{rgb}{0.58,0,0.82}
\definecolor{backcolour}{rgb}{0.95,0.95,0.92}
\graphicspath{{Images/}}
\begin{document}
*This document is written in \LaTeX, by Aditya Raj (2204126/ECE-B), for raw \LaTeX code click \href{https://github.com/hexronuspi/Latex-hackslash-Gray/blob/main/main.tex}{\textcolor{blue}{here}}.
\begin{center}
\textbf{\Large Team Gray Interface (ML Team) Induction Tasks}
\end{center}
\section{Problem Statement}

You will be working on a USA housing price dataset. The features on it will be:

\begin{itemize}
\item Avg. area income - Average income of the family living in that area.
\item Avg. area house age - Average age of the houses in a particular locality.
\item Avg. area number of rooms - Average number of rooms in the houses in that area.
\item Avg. area number of bedrooms - Average number of bedrooms.
\item Area Population.
\item Price (Prediction target).
\item Address - address of that locality.
\end{itemize}
To-Do:
\begin{itemize}
\item Prepare a detailed python notebook using Linear Regression for predicting the price of houses in any locality.
\item Import all the required libraries.
\item Load and pre-process the dataset using the pandas library.
\item Print the column data types.
\item Visualize the dataset (plot the graphs whenever needed).
\item Train the model on the dataset and print error and RMSE (Root Mean Squared Error).
\end{itemize}
\vspace{12\baselineskip}
\section{Solution}
We will use a dataset of US house price data for this research. The dataset includes a variety of characteristics, including the average area income, average area home age, average area number of rooms, average area number of beds, average area population, average area price (prediction goal), and average area address. Our objective is to create a linear regression model that can forecast property prices wherever.

\vspace{1\baselineskip}
The following actions will be taken in order to accomplish this goal:

\begin{enumerate}
\item Import all the required libraries.
\item Load and pre-process the dataset using the pandas library.
\item Print the column data types.
\item Visualize the dataset using graphs whenever needed.
\item Train the model on the dataset using linear regression.
\item Evaluate the model by printing the error and RMSE (Root Mean Squared Error).
\end{enumerate}

To assess the model's effectiveness, we will use the RMSE measure. The performance of the model is improved by a decreased RMSE. Our ultimate objective is to create a model that, using the provided information, can precisely estimate the price of homes in any location.
\vspace{1\baselineskip}

The program was checked with different values of testsize and 
randomstate, \textcolor{red}{for randomstate = 45 and testsize = 0.40 the minimum of the RMSE was determied.(The minimum came out to be 	$\approx$ 0, when 'Price' was dropped during calculation')}
\begin{table}[h]
\centering
\caption{Test RMSE Cases}
\label{tab:test-rmse}
\begin{tabular}{|c|c|c|}
\hline
\textcolor{red}{\textbf{testsize}} & \textcolor{red}{\textbf{random state}} & \textcolor{red}{\textbf{RMSE}} \\ \hline
0.30 & 42 & 100325.1651816 \\ \hline
0.30 & 45 & 99843.8350706 \\ \hline
0.40 & 45 & 99545.2616282 \\ \hline
0.40 & 45 & 	$\approx$ 0(Zero) :: Price was "dropped" while calculation \\ \hline
\end{tabular}
\end{table}


\vspace{21\baselineskip}

Google Colab Notebook (Drive) \href{https://drive.google.com/drive/folders/1SFkQtmzvhyU-O0_g99m80oqBAWkEj22J?usp=sharing}{\textcolor{blue}{(link)}}.The Google drive link can be found\href{https://drive.google.com/drive/folders/1oUOGVvu0X8MpYs0S50cgQ-SZPvm9ikcR?usp=sharing}{\textcolor{blue}{(here)}}.
\begin{minted}{python}
#importing the libraries

import pandas as pd
#Pandas is a library for analysing and manipulating data.

import numpy as np
#Python's numpy package is used for scientific computing.

import seaborn as sns
#A data visualisation library based on matplotlib is called seaborn.

import matplotlib.pyplot as plt
#Python's matplotlib package is used to build visualisations.

from sklearn.model_selection import train_test_split
#The sklearn.model selection module's train test split
#method divides the data into training and testing sets.

from sklearn.linear_model import LinearRegression
#To develop a linear regression model, a class called
#LinearRegression from the sklearn.linear model module will
#be utilised.

from sklearn.metrics import mean_squared_error
#The RMSE will be determined using the sklearn.metrics
#module's mean squared error function.

# Load the dataset
housing_data = pd.read_csv('USA_Housing.csv')
#In this step, we use the read csv function to import the dataset into the DataFrame.

# Check the column data types
print("Column data types:")
print(housing_data.dtypes)

# Visualize the dataset
print("Visualizing the dataset...")
sns.pairplot(housing_data)
plt.show()
#In this phase, the pairplot function from the Seaborn
#library is used to show the dataset. Every pair of
#variables in the dataset has a scatterplot created by this
#function. The plot is shown using the plt.show() method.



housing_data.info()

# Check for missing values
print(housing_data.isnull().sum())

housing_data.head()

# Train the model on the dataset and print error and RMSE 
X = housing_data.drop([ 'Address'], axis=1)
#I dropped the "Price" and "Address" columns from the dataset and
#used the remaining columns as input features to train the linear #regression model. 
#The reason for dropping the "Price" column is that it is the target variable that we
#are trying to predict, and we don't want it to be included as 
#a feature in the model. The reason for dropping the "Address" column is that it is
#a non-numeric categorical variable that cannot be used directly
#as a feature in a linear regression model.

#since we are working on linear regression, the representaion
#[F=A+B(x1)+C(x2)+D(x3)+........], this model will fall apart for more
#number of variables and if the input values are not 
#linearly placed in the dataset(which is a hypothetical senario by def.),
#so using more and more variables will affect the output severely.
y = housing_data['Price']
X_train, X_test, y_train, y_test = train_test_split(X, y, test_size=0.40, random_state=45)

print("Training the model...")
regressor = LinearRegression()
regressor.fit(X_train, y_train)

print('Model Coefficients:', regressor.coef_)

print("Making predictions on the test set...")
y_pred = regressor.predict(X_test)

rmse = np.sqrt(mean_squared_error(y_test, y_pred))
print('Root Mean Squared Error:', rmse)

plt.scatter(y_test, y_pred)
plt.xlabel('Actual Prices')
plt.ylabel('Predicted Prices')
plt.show()
\end{minted}
\vspace{1\baselineskip}
\includegraphics{4.png}
\includegraphics{1.png}

\vspace{8\baselineskip}
This is when 'Price', and 'Address' were dropped.

\includegraphics{3.png}

\vspace{16\baselineskip}

This is when 'Address' was dropped.

\includegraphics{2.png}

\end{document}

